%\documentclass{beamer}
\documentclass[handout]{beamer}

\usepackage[brazilian]{babel}
\usepackage[utf8]{inputenc}
\usepackage{graphicx}
\usepackage{fontenc}
\usepackage{listings}
\usepackage{verbatim}
\usepackage{pxfonts}
\usepackage{graphicx}
\usetheme{Amsterdam}

\title{Grandes Migrações}
\subtitle{Passando de Qualquer Plataforma \\
  Para o WordPress}
\author{Vinicius Massuchetto}

\lstset{%
  breakatwhitespace,
  columns=fullflexible,
  keepspaces,
  breaklines,
  tabsize=2,
  showstringspaces=false,
  extendedchars=true,
  basicstyle=\footnotesize\ttfamily,
  frame=leftline}

\begin{document}

\frame{\titlepage}

\section{Introdução}
\subsection{Introdução}

\begin{frame}{Download}
  \begin{center}
    Esta apresentação está disponível em:

    \url{https://github.com/vmassuchetto/wp-migrations}

    branch:intercon
  \end{center}
\end{frame}

\begin{frame}{Sobre o Que Falaremos}
\begin{itemize}
  \item Por que tratar do assunto
  \item Migrações simples e suportadas
  \item Estratégias de migração de outras plataformas
\end{itemize}
\end{frame}

\section{Motivação}
\subsection{Motivação}

\begin{frame}{Por que falar sobre migrações com o cliente?}
\begin{itemize}
  \pause \item Indexação de conteúdo
  \pause \item Manutenção de usabilidade
  \pause \item Reestruturação do conteúdo
\end{itemize}
\end{frame}

\begin{frame}{Por que falar sobre migrações com a equipe de desenvolvimento?}
\begin{itemize}
  \pause \item Análise de complexidade
  \pause \item Análise de correlação e criação de estruturas
  \pause \item Resolução de velhos problemas
  \pause \item Definição de estratégias
  \pause \item Delegação de tarefas
  \pause \item Elaboração de manuais
  \pause \item Definição do tempo de projeto
\end{itemize}
\end{frame}

\begin{frame}{E na verdade, as migrações são..}
\begin{itemize}
  \pause \item Uma etapa de projeto que poderia ser melhor discutida
  \pause \item Uma das partes mais importantes da implantação de projetos web
\end{itemize}
\end{frame}

\section{Migrações Simples}
\subsection{Migrações Simples}

\begin{frame}
\frametitle{O que são migrações para o WordPress?}
\begin{itemize}
  \pause \item WordPress $\rightarrow$ WordPress
  \pause \item Plataformas Suportadas $\rightarrow$ WordPress
  \pause \item Outras Plataformas $\rightarrow$ WordPress
\end{itemize}
\end{frame}

\begin{frame}{WordPress $\rightarrow$ WordPress}
\begin{itemize}
  \pause \item Geralmente tranquila quando feita corretamente
  \pause \item Requere menos processamento e recuperação de conteúdos
               externos
  \pause \item Casos:
    \begin{itemize}
      \pause \item Quando a URL não muda
      \pause \item Quando a URL muda
    \end{itemize}
\end{itemize}
\end{frame}

\begin{frame}{WordPress $\rightarrow$ WordPress: Mesma URL}
\begin{itemize}
  \pause \item Copiar a base
  \pause \item Modificar o \texttt{wp-config.php}
\end{itemize}
\end{frame}

\begin{frame}[fragile]{WordPress $\rightarrow$ WordPress: Mesma URL}
  \lstinputlisting{./code/wp-wp-dump.sh}
\end{frame}

\begin{frame}{WordPress $\rightarrow$ WordPress: URLs Diferentes}
\begin{itemize}
  \pause \item No WordPress, muitas URLs ficam persistentes no banco de dados
  \pause \item Buscar e substituir não resolve
\end{itemize}
\end{frame}

\begin{frame}[fragile]{WordPress $\rightarrow$ WordPress: URLs Diferentes}
  \lstinputlisting{./code/wp-wp-meta.txt}
  \pause
  \lstinputlisting{./code/wp-wp-meta-serialized.txt}
\end{frame}

\begin{frame}[fragile]{WordPress $\rightarrow$ WordPress: URLs Diferentes}
  \lstinputlisting{./code/wp-wp-meta-serialized-replaced.txt}
  \pause
  \lstinputlisting{./code/wp-wp-meta-key.php}
  \pause
  \lstinputlisting{./code/wp-wp-meta-serialized-rep-dump.txt}
\end{frame}

\begin{frame}{WordPress $\rightarrow$ WordPress: URLs Diferentes}
\begin{itemize}
  \item Para mudança de URLs deve-se fazer a substituição
        adequadamente, via plugin ou script
  \pause \item Exemplos:
  \begin{itemize}
    \item Scripts: \texttt{searchreplacedb2.php}, \texttt{migra\_bd.php}
    \item Plugins: WordPress Move, Search and Replace, WP Migrate Tool
  \end{itemize}
\end{itemize}
\end{frame}

\begin{frame}[fragile]{WordPress $\rightarrow$ WordPress: URLs Diferentes}
  \lstinputlisting{./code/wp-wp-dump.sh}
  e..
\end{frame}

\begin{frame}[fragile]{WordPress $\rightarrow$ WordPress: URLs Diferentes}
  \lstinputlisting{./code/wp-wp-dump-replace-01.sh}
  \pause
  \lstinputlisting{./code/wp-wp-dump-replace-02.sh}
  \pause
  \lstinputlisting{./code/wp-wp-dump-replace-03.sh}
  \pause
  \lstinputlisting{./code/wp-wp-dump-replace-04.sh}
\end{frame}

\begin{frame}[fragile]{WordPress $\rightarrow$ WordPress: URLs Diferentes}
  \lstinputlisting{./code/wp-wp-serialized-replaced.txt}
  \pause
  \lstinputlisting{./code/wp-wp-meta-key.php}
\end{frame}

\begin{frame}[fragile]{WordPress $\rightarrow$ WordPress: URLs Diferentes}
  \lstinputlisting{./code/wp-wp-meta-serialized-replaced-dump.txt}
\end{frame}

\begin{frame}{WordPress $\rightarrow$ Suportados}
\begin{itemize}
  \pause
  \item Blogger
  \item LiveJournal
  \item Movable Type
  \item RSS
  \item Tumblr
  \item Plugins \ldots
\end{itemize}
\end{frame}

\section{Migrações Complexas}
\subsection{Migrações Complexas}

\begin{frame}{Tópicos a Serem Levados em Conta}
  \begin{itemize}
    \pause \item Tecnologia
    \pause \item Estrutura
    \pause \item Referências e relações internas
    \pause \item Conteúdo
    \pause \item Mídias
    \pause \item URLs
  \end{itemize}
\end{frame}

\begin{frame}{Tecnologias Empregadas}
  \begin{itemize}
    \pause \item Configuração dos servidores\pause \\
                 (safe mode, parâmetros de compilação)
    \begin{itemize}
      \pause \item \texttt{ini\_set( 'memory\_limit', -1)}
      \pause \item \texttt{set\_time\_limit( 0 )}
      \pause \item ou.. \pause ajax recursivo
    \end{itemize}
    \pause \item Modo de obtenção de dados (socket, webservice, csv)
    \pause \item Linguagem a serem escritos os scripts de migração
    \pause \item Preferência: \pause PHP\pause, MySQL\pause, de dentro
                 do WordPress
  \end{itemize}
\end{frame}

\begin{frame}{Tecnologias Empregadas}
  \begin{itemize}
    \pause \item Migrar através do próprio WordPress:
    \begin{itemize}
      \pause \item Facilidade e padronização de manipulação dos dados
      \pause \item Garantia de integridade
    \end{itemize}
  \end{itemize}
\end{frame}

\begin{frame}{Tecnologias Empregadas: wpdb}
  \begin{itemize}
    \pause \item Classe \texttt{wpdb}
    \begin{itemize}
      \pause \item \texttt{query()}
      \pause \item \texttt{get\_results()}
      \pause \item \texttt{get\_var()}
    \end{itemize}
  \end{itemize}
\end{frame}

\begin{frame}{Tecnologias Empregadas: wpdb}
  \lstinputlisting{./code/tec-wpdb.php}
\end{frame}

\begin{frame}{Tecnologias Empregadas: Abstração de Banco}
    \begin{itemize}
    \pause \item Funções de relação com o banco de dados:
    \begin{itemize}
      \pause \item \texttt{wp\_insert\_post()}
      \pause \item \texttt{wp\_insert\_term()}
      \pause \item \texttt{wp\_set\_post\_terms()}
      \pause \item \texttt{wp\_insert\_attachment()}
      \pause \item \texttt{wp\_update\_attachment\_metadata()}
      \pause \item \texttt{update\_post\_meta()}
    \end{itemize}
  \end{itemize}
\end{frame}

\begin{frame}{Tecnologias Empregadas: Exemplo de Rotina}
  \lstinputlisting{./code/tec-routine.php}
\end{frame}

\begin{frame}{Estrutura}
  \begin{center}
    \pause \includegraphics[height=0.8\textheight,natwidth=1079,natheight=1089]{./img/wp-tables.png}
  \end{center}
\end{frame}

\begin{frame}{Estrutura: Mapeamento}
  \begin{itemize}
    \pause \item Identificação das entidades \pause (post types)
    \pause \item Identificação dos atributos \pause (custom fields)
    \pause \item Identificação das relações \pause (taxonomias ou relações de plugins)
  \end{itemize}
\end{frame}

\begin{frame}{Estrutura: Mapeamento}
  Exemplo: Linha de tabela \texttt{estabelecimento}
  \begin{itemize}
    \item \texttt{(int) id}
    \item \texttt{(str) nome}
    \item \texttt{(str) descricao}
    \item \texttt{(str) rua}
    \item \texttt{(str) bairro}
    \item \texttt{(fk) cidade}
    \item \texttt{(fk) estado}
    \item \texttt{(fk) tipo}
    \item \texttt{(fk) foto}
  \end{itemize}
\end{frame}

\begin{frame}{Estrutura: Mapeamento}
  \begin{itemize}
    \pause \item Post type \texttt{estabelecimento}
    \begin{itemize}
      \item \texttt{post\_title: nome}
      \item \texttt{post\_content: descricao}
    \end{itemize}
    \pause \item Chaves meta
    \begin{itemize}
      \item \texttt{meta\_key: rua}
      \item \texttt{meta\_key: bairro}
      \item \texttt{meta\_key: \_id}
    \end{itemize}
    \pause \item Taxonomia
    \begin{itemize}
      \item Tipo de Estabelecimento
    \end{itemize}
    \pause \item Anexo
    \begin{itemize}
      \item Foto
    \end{itemize}
  \end{itemize}
\end{frame}

\begin{frame}{Estrutura: Mapeamento}
  \begin{itemize}
    \item \texttt{cidade (??)}
    \item \texttt{estado (??)}
  \end{itemize}
  \pause Soluções possíveis:
  \begin{itemize}
    \item Taxonomia
    \item \texttt{post\_parent}
    \item Relação via custom fields
    \item Relação via plugin \\
          (Advanced Custom Fields, Posts2Posts)
    \item Outras tecnologias
  \end{itemize}
\end{frame}

\begin{frame}{Mudança de tecnologia}
  \begin{center}
    \pause \includegraphics[height=0.8\textheight,natwidth=800,natheight=731]{./img/catracalivre.png}
  \end{center}
\end{frame}

\begin{frame}{Formatação: Basicamente o conteúdo}
  \begin{itemize}
    \item Elaboração de inventário de conteúdo
    \item Conversão de caminhos
    \item Conversão de embeds em shortcodes
    \item Conversão de estruturas HTML editáveis
    \item Remoção de scripts
  \end{itemize}
\end{frame}

\begin{frame}{Tratamento de conteúdo}
  \begin{itemize}
    \pause \item Funções de tratamento:
    \begin{itemize}
      \pause \item \texttt{remove\_accents()}
      \pause \item \texttt{sanitize\_title()}
      \pause \item \texttt{normalize\_whitespace()}
      \pause \item \texttt{make\_clickable()}
      \pause \item \texttt{capital\_P\_dangit()}
    \end{itemize}
  \end{itemize}
\end{frame}

\begin{frame}{Formatação: Capturando Links}
  \lstinputlisting{./code/ns-migrating-links.php}
\end{frame}

\begin{frame}{Migração de Mídias}
  \lstinputlisting{./code/ns-migrating-media-01.php}
\end{frame}

\begin{frame}{Migração de Mídias}
  \lstinputlisting{./code/ns-migrating-media-02.php}
\end{frame}

\begin{frame}{Migração de Mídias}
  \lstinputlisting{./code/ns-migrating-media-03.php}
\end{frame}

\begin{frame}{Migração de Mídias: Serviço Dinâmico}
  \begin{itemize}
    \item Ao invés do download pode-se fazer o serviço dinâmico de mídias.
          Veja o arquivo \\
          \texttt{wp-includes/ms-files.php}
  \end{itemize}
\end{frame}

\begin{frame}{Migração de Mídias: Serviço Dinâmico}
  \lstinputlisting{./code/ns-migrating-media-dynamic-rr.txt}
\end{frame}

\begin{frame}{Migração de Mídias: Serviço Dinâmico}
  \lstinputlisting{./code/ns-migrating-media-dynamic.php}
\end{frame}

\section{Considerações Finais}
\subsection{Considerações Finais}

\begin{frame}{Considerações Finais}
  \begin{itemize}
    \item Migrações devem definitivamente ser incluídas como
          componentes de projeto
    \item O WordPress oferece um bom conjunto de ferramentas
          para se trazer dados para dentro de sua estrutura.
    \item É cabível o desenvolvimento de plugins específicos
          para a migração de diferentes estruturas
  \end{itemize}
\end{frame}

\end{document}
